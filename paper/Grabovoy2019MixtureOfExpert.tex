\documentclass[12pt, twoside]{article}
\usepackage[utf8]{inputenc}
\usepackage[english,russian]{babel}

\usepackage{amsthm}
\usepackage{a4wide}
\usepackage{graphicx}
\usepackage{caption}
\usepackage{amssymb}
\usepackage{amsmath}
\usepackage{mathrsfs}
\usepackage{euscript}
\usepackage{graphicx}
\usepackage{subfig}
\usepackage{caption}
\usepackage{color}
\usepackage{bm}
\usepackage{tabularx}
\usepackage{adjustbox}


\usepackage[toc,page]{appendix}

\usepackage{comment}
\usepackage{rotating}

\DeclareMathOperator*{\argmax}{arg\,max}
\DeclareMathOperator*{\argmin}{arg\,min}

\newtheorem{theorem}{Теорема}
\newtheorem{lemma}[theorem]{Лемма}
\newtheorem{definition}{Определение}[section]

\numberwithin{equation}{section}

\newcommand*{\No}{No.}
\begin{document}

\title{\bf Смесь экспертов\thanks{Работа выполнена при поддержке РФФИ и правительства РФ.}}
\date{}
\author{}
\maketitle

\begin{center}
\bf
А.\,В.~Грабовой\footnote{Московский физико-технический институт, grabovoy.av@phystech.edu}, В.\,В.~Стрижов\footnote{Московский физико-технический институт, strijov@ccas.ru}

\end{center}

{\centering\begin{quote}
\textbf{Аннотация:} 


\smallskip
\textbf{Ключевые слова}: ; .

\smallskip
\textbf{DOI}: 00.00000/00000000000000
\end{quote}
}

\section{Введение}

\section{Постановка задачи смеси экспертов}

Задана выборка:
\begin{equation}
\label{eq:st:1}
\begin{aligned}
\textbf{X} \in \mathbb{R}^{N \times n},
\end{aligned}
\end{equation}
где~$N$~---~количество объектов в выборке, а~$n$~---~размерность признакового пространства.

\begin{definition}
Смесь экспертов~---~мультимодель, определяющая правдоподобие каждой $\pi_k$ каждой модели $\textbf{f}_k$ на объекте $\textbf{x}$ на основе его признакового опсиания.

\begin{equation}
\label{eq:st:2}
\begin{aligned}
\hat{\textbf{f}} = \sum_{k=1}^{K}\pi_{k}\textbf{f}_k, \qquad \pi_{k}\left(\textbf{x}, \textbf{V}\right):\mathbb{R}^{2\times n} \to [0, 1], \qquad \sum_{k=1}^{K}\pi_{k}\left(\textbf{x}, \textbf{V}\right) = 1
\end{aligned}
\end{equation}
где~$\textbf{f}$~---~мультимодель, а $\textbf{f}_k$ является некоторой моделью, $\pi_k$~---~параметрическая модель.
\end{definition}

\subsection{Общий случай}

Правдоподобие модели:
\begin{equation}
\label{eq:st:3}
\begin{aligned}
p\left(\textbf{y}, \textbf{W}, \textbf{Z}|\textbf{X}, \textbf{V}, \textbf{A}\right) &= \prod_{i=1}^{N}\prod_{k=1}^{K}\left[\pi_{k}\left(\textbf{x}_i,\textbf{V}\right)p_k\left(y_i|\textbf{w}_{k}, \textbf{x}_i\right)\right]^{z_{ik}}\prod_{k=1}^{K}p^{k}\left(\textbf{w}_{k}|\textbf{A}_{k}\right)
\end{aligned}
\end{equation}

Логарифм правдоподобия модели:
\begin{equation}
\label{eq:st:4}
\begin{aligned}
\log p\left(\textbf{y}, \textbf{W}, \textbf{Z}|\textbf{X}, \textbf{V}, \textbf{A}\right) &= \sum_{i=1}^{N}\sum_{k=1}^{K}z_{ik}\left[\log\pi_k\left(\textbf{x}_i, \textbf{V}\right) + \log p_k\left(y_i|\textbf{w}_{k}, \textbf{x}_{i}\right)\right] +\\
&+ \sum_{k=1}^{K}\log p^{k}\left(\textbf{w}_k|\textbf{A}_k\right)
\end{aligned}
\end{equation}

\paragraph{E-step}~

Найдем $q\left(\textbf{Z}\right)$:
\begin{equation}
\label{eq:st:5}
\begin{aligned}
\log q\left(\textbf{Z}\right) &= \mathsf{E}_{q/\textbf{Z}}\log p\left(\textbf{y}, \textbf{W}, \textbf{Z}|\textbf{X}, \textbf{V}, \textbf{A}\right) \\
p\left(z_{ik} = 1\right) &= \frac{\exp\left(\log\pi_{k}\left(\textbf{x}_{i}, \textbf{V}\right) + \mathsf{E}\log p_{k}\left(y_i|\textbf{w}_k, \textbf{A}_k\right)\right)}{\sum_{k'=1}^{K}\exp\left(\log\pi_{k'}\left(\textbf{x}_{i}, \textbf{V}\right) + \mathsf{E}\log p_{k'}\left(y_i|\textbf{w}_k', \textbf{A}_k'\right)\right)}
\end{aligned}
\end{equation}

Найдем $q\left(\textbf{W}\right)$:
\begin{equation}
\label{eq:st:6}
\begin{aligned}
\log q\left(\textbf{W}\right) &= \mathsf{E}_{q/\textbf{W}}\log p\left(\textbf{y}, \textbf{W}, \textbf{Z}|\textbf{X}, \textbf{V}, \textbf{A}\right) \\
&= \sum_{i=1}^{N}\sum_{k=1}^{K}\mathsf{E}z_{ik}\left[\log\pi_{k}\left(\textbf{x}_{i, \textbf{V}}\right) + \log p_{k}\left(y_{i}\left(y_{i}|\textbf{w}_{k}, \textbf{x}_i\right)\right)\right] + \\
&+ \sum_{k=1}^{K}\log p^{k}\left(\textbf{w}_{k}, \textbf{A}_{k}\right)
\end{aligned}
\end{equation}

\paragraph{M-step}~
\begin{equation}
\label{eq:st:7}
\begin{aligned}
\mathsf{E}_{q} p\left(\textbf{y}, \textbf{W}, \textbf{Z}|\textbf{X}, \textbf{V}, \textbf{A}\right) &= \mathcal{F}\left(\textbf{V}, \textbf{A}\right) \\
\mathcal{F}\left(\textbf{V}, \textbf{A}\right) &= \sum_{i=1}^{N}\sum_{k=1}^{K}\mathsf{E}z_{ik}\left[\log\pi_k\left(\textbf{x}_i, \textbf{V}\right) + \mathsf{E}\log p_{k}\left(y_i|\textbf{w}_{k}, \textbf{x}_{i}\right)\right] \\
&+ \sum_{k=1}^{K}\mathsf{E}\log p_{k}\left(\textbf{w}_{k}|\textbf{A}_k\right)
\end{aligned}
\end{equation}

Найдем $\textbf{A}$ из условия:
\begin{equation}
\label{eq:st:8}
\begin{aligned}
\frac{\partial \mathcal{F}\left(\textbf{V}, \textbf{A}\right)}{\partial \textbf{A}^{-1}} = 0 
\end{aligned}
\end{equation}

Найдем $\textbf{V}$:

Аналитически решение не ищется, поэтому воспользуемся градиентным спуском для максимизации правдоподобия модели:
\begin{equation}
\label{eq:st:9}
\begin{aligned}
\textbf{V}^{j+1} &= \textbf{V}^{j} + \alpha\frac{\partial \mathcal{F}\left(\textbf{W}, \textbf{V}^{j}, \beta\right)}{\partial \textbf{V}} 
\end{aligned}
\end{equation}

\subsection{Случай $p_k\left(y_i|\textbf{X}, \textbf{w}_k\right) = \text{N}\left(y_i|\textbf{w}_{k}^{\mathsf{T}}\textbf{x}_i, \beta^{-1}\right)$}

\paragraph{E-step}~

Для нахождения $q\left(\textbf{Z}\right)$ требуется:
\begin{equation}
\label{eq:st:10}
\begin{aligned}
\mathsf{E}\log p_{k}\left(y_{i}|\textbf{w}_k, \textbf{A}_k\right) &= -\frac{\beta}{2}\left(y_{i}^{2}-2y_{i}\textbf{x}_{i}^{\mathsf{T}}\mathsf{E}\textbf{w}_{k}+\textbf{x}_{i}^{\mathsf{T}}\left(\mathsf{E}\textbf{w}_{k}\textbf{w}_{k}^{\mathsf{T}}\right)\textbf{x}_{i}\right) + \frac{1}{2}\left(\log\beta - \log2\pi\right)\\
\end{aligned}
\end{equation}

Найдем $q\left(\textbf{w}_{k}\right)$:
\begin{equation}
\label{eq:st:11}
\begin{aligned}
\log q\left(\textbf{w}_{k}\right) &\propto -\frac{1}{2}\left(\textbf{w}_{k}^{\mathsf{T}}\textbf{A}_{k}^{-1}\textbf{w}_{k} +\beta \sum_{i=1}^{N}\mathsf{E}z_{ik}\left[y_{i} - \textbf{w}_{k}^{\mathsf{T}}\textbf{x}_{i}\right]^{2}\right) \propto\\
&\propto -\frac{1}{2}\left(\textbf{w}_{k} - \textbf{m}_{k}\right)^{\mathsf{T}}\textbf{B}_{k}^{-1}\left(\textbf{w}_{k} - \textbf{m}_{k}\right)
\end{aligned}
\end{equation}
где введены обозначения:
\begin{equation}
\label{eq:st:12}
\begin{aligned}
\textbf{B}_{k} = \left(\textbf{A}_{k}^{-1} + \beta\sum_{i=1}^{N}\textbf{x}_{i}\textbf{x}_{i}^{\mathsf{T}}\mathsf{E}z_{ik}\right)^{-1}, \qquad \textbf{m}_{k} = \beta\textbf{B}_{k}\left(\sum_{i=1}^{N}\textbf{x}_{i}y_{i}\mathsf{E}z_{ik}\right).
\end{aligned}
\end{equation}

\paragraph{M-step}~
\begin{equation}
\label{eq:st:13}
\begin{aligned}
\mathsf{E}_{q} p\left(\textbf{y}, \textbf{W}, \textbf{Z}|\textbf{X}, \textbf{V}, \textbf{A}, \beta\right) &= \mathcal{F}\left(\textbf{V}, \textbf{A}, \beta\right). \\
\mathsf{E}\log p_{k}\left(y_{i}|\textbf{w}_{k}, \textbf{x}_{i}\right) &= -\frac{\beta}{2}\left(y_{i}^{2}-2y_{i}\textbf{x}_{i}^{\mathsf{T}}\mathsf{E}\textbf{w}_{k}+\textbf{x}_{i}^{\mathsf{T}}\left(\mathsf{E}\textbf{w}_{k}\textbf{w}_{k}^{\mathsf{T}}\right)\textbf{x}_{i}\right) + \frac{1}{2}\left(\log\beta - \log2\pi\right)\\
\mathsf{E}\log p_{k}\left(\textbf{w}_{k}|\textbf{A}_{k}\right) &= -\frac{1}{2}\mathsf{E}\textbf{w}_{k}^{\mathsf{T}}\textbf{A}_{k}^{-1}\textbf{w}_{k} + \frac{1}{2}\log\det\textbf{A}_{k}^{-1} - \frac{n}{2}\log2\pi
\end{aligned}
\end{equation}

Найдем $\beta$:
\begin{equation}
\label{eq:st:14}
\begin{aligned}
\frac{\partial \mathcal{F}\left(\textbf{V}, \textbf{A}, \beta\right)}{\partial \beta} &= \sum_{k=1}^{K}\sum_{i=1}^{N}\left(\frac{1}{\beta}\mathsf{E}z_{ik}-\frac{1}{2}\mathsf{E}z_{ik}\left[y_{i}^{2}-2y_{i}\textbf{x}_{i}^{\mathsf{T}}\mathsf{E}\textbf{w}_{k}+\textbf{x}_{i}^{\mathsf{T}}\textbf{w}_{k}\textbf{w}_{k}^{\mathsf{T}}\textbf{x}_{i}\right]\right) = 0\\
\frac{1}{\beta}&=\frac{1}{N}\sum_{i=1}^{N}\sum_{k=1}^{K}\left[y_{i}^{2}-2y_{i}\textbf{x}_{i}^{\mathsf{T}}\mathsf{E}\textbf{w}_{k} + \textbf{x}_{i}^{\mathsf{T}}\mathsf{E}\textbf{w}_{k}\textbf{w}_{k}^{\mathsf{T}}\textbf{x}_{i}\right]\mathsf{E}z_{ik}
\end{aligned}
\end{equation}

Найдем $\textbf{A}$:
\begin{equation}
\label{eq:st:15}
\begin{aligned}
\frac{\partial \mathcal{F}\left(\textbf{V}, \textbf{A}, \beta\right)}{\partial \textbf{A}^{-1}} &= \frac{1}{2}\textbf{A}_{k} - \frac{1}{2}\mathsf{E}\textbf{w}_{k}\textbf{w}_{k}^{\mathsf{T}} = 0 \Rightarrow \textbf{A}_{k}^{new} = \text{diag}\left(\mathsf{E}\textbf{w}_{k}\textbf{w}_{k}^{\mathsf{T}}\right)
\end{aligned}
\end{equation}

Найдем $\textbf{V}$:

Аналитически решение не ищется, поэтому воспользуемся градиентным спуском для максимизации правдоподобия модели:
\begin{equation}
\label{eq:st:16}
\begin{aligned}
\textbf{V}^{j+1} &= \textbf{V}^{j} + \alpha\frac{\partial \mathcal{F}\left(\textbf{W}, \textbf{V}^{j}, \beta\right)}{\partial \textbf{V}} 
\end{aligned}
\end{equation}

\subsection{Случай $p_k\left(y_i|\textbf{X}, \textbf{w}_k\right) = \text{Exp}\left(y_{i}-\textbf{w}_{k}^{\mathsf{E}}\textbf{x}_{i}|\frac{\beta}{2}\right)$}

\paragraph{E-step}~

Для нахождения $q\left(\textbf{Z}\right)$ требуется:
\begin{equation}
\label{eq:st:17}
\begin{aligned}
\mathsf{E}\log p_{k}\left(y_{i}|\textbf{w}_k, \textbf{A}_k\right) &= \log\frac{\beta}{2} -\frac{\beta}{2}\left(y_{i}-\textbf{w}_{k}^{\mathsf{T}}\textbf{x}_{i}\right)
\end{aligned}
\end{equation}

Найдем $q\left(\textbf{w}_{k}\right)$:
\begin{equation}
\label{eq:st:18}
\begin{aligned}
\log q\left(\textbf{w}_{k}\right) &\propto -\frac{1}{2}\left(\textbf{w}_{k}^{\mathsf{T}}\textbf{A}_{k}^{-1}\textbf{w}_{k} +\frac{\beta}{2} \sum_{i=1}^{N}\mathsf{E}z_{ik}\left[y_{i} - \textbf{w}_{k}^{\mathsf{T}}\textbf{x}_{i}\right]\right) \propto\\
&\propto -\frac{1}{2}\left(\textbf{w}_{k} - \textbf{m}_{k}\right)^{\mathsf{T}}\textbf{A}_{k}^{-1}\left(\textbf{w}_{k} - \textbf{m}_{k}\right)
\end{aligned}
\end{equation}
где введено обозначения:
\begin{equation}
\label{eq:st:19}
\begin{aligned}
\textbf{m}_{k} = \frac{\beta}{2}\textbf{A}_{k}\left(\sum_{i=1}^{N}\textbf{x}_{i}\mathsf{E}z_{ik}\right).
\end{aligned}
\end{equation}

\paragraph{M-step}~
\begin{equation}
\label{eq:st:20}
\begin{aligned}
\mathsf{E}_{q} p\left(\textbf{y}, \textbf{W}, \textbf{Z}|\textbf{X}, \textbf{V}, \textbf{A}, \beta\right) &= \mathcal{F}\left(\textbf{V}, \textbf{A}, \beta\right). \\
\mathsf{E}\log p_{k}\left(y_{i}|\textbf{w}_{k}, \textbf{x}_{i}\right) &= \log\frac{\beta}{2} -\frac{\beta}{2}\left(y_{i}-\textbf{w}_{k}^{\mathsf{T}}\textbf{x}_{i}\right)\\
\mathsf{E}\log p_{k}\left(\textbf{w}_{k}|\textbf{A}_{k}\right) &= -\frac{1}{2}\mathsf{E}\textbf{w}_{k}^{\mathsf{T}}\textbf{A}_{k}^{-1}\textbf{w}_{k} + \frac{1}{2}\log\det\textbf{A}_{k}^{-1} - \frac{n}{2}\log2\pi
\end{aligned}
\end{equation}

Найдем $\beta$:
\begin{equation}
\label{eq:st:21}
\begin{aligned}
\frac{\partial \mathcal{F}\left(\textbf{V}, \textbf{A}, \beta\right)}{\partial \beta} &= \sum_{k=1}^{K}\sum_{i=1}^{N}\mathsf{E}z_{ik}\left[\frac{1}{\beta} - \frac{y_i - \textbf{w}_{k}^{\mathsf{T}}\textbf{x}_{i}}{2}\right] = 0\\
\frac{1}{\beta} &= \frac{1}{2N}\sum_{i=1}^{N}\sum_{k=1}^{K}\left(y_{i}-\textbf{w}_{k}^{\mathsf{T}}\textbf{x}_{i}\right)\mathsf{E}z_{ik}
\end{aligned}
\end{equation}

Найдем $\textbf{A}$:
\begin{equation}
\label{eq:st:22}
\begin{aligned}
\frac{\partial \mathcal{F}\left(\textbf{V}, \textbf{A}, \beta\right)}{\partial \textbf{A}^{-1}  }&= \frac{1}{2}\textbf{A}_{k} - \frac{1}{2}\mathsf{E}\textbf{w}_{k}\textbf{w}_{k}^{\mathsf{T}} = 0 \Rightarrow \textbf{A}_{k}^{new} = \text{diag}\left(\mathsf{E}\textbf{w}_{k}\textbf{w}_{k}^{\mathsf{T}}\right)
\end{aligned}
\end{equation}


Найдем $\textbf{V}$:

Аналитически решение не ищется, поэтому воспользуемся градиентным спуском для максимизации правдоподобия модели:
\begin{equation}
\label{eq:st:23}
\begin{aligned}
\textbf{V}^{j+1} &= \textbf{V}^{j} + \alpha\frac{\partial \mathcal{F}\left(\textbf{W}, \textbf{V}^{j}, \beta\right)}{\partial \textbf{V}} 
\end{aligned}
\end{equation}


\section{Вычислительный эксперимент}

\section{Заключение}

\begin{thebibliography}{99}

\end{thebibliography}

\end{document}

