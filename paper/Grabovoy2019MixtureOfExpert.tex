\documentclass[12pt, twoside]{article}
\usepackage[utf8]{inputenc}
\usepackage[english,russian]{babel}

\usepackage{amsthm}
\usepackage{a4wide}
\usepackage{graphicx}
\usepackage{caption}
\usepackage{amssymb}
\usepackage{amsmath}
\usepackage{mathrsfs}
\usepackage{euscript}
\usepackage{graphicx}
\usepackage{subfig}
\usepackage{caption}
\usepackage{color}
\usepackage{bm}
\usepackage{tabularx}
\usepackage{adjustbox}


\usepackage[toc,page]{appendix}

\usepackage{comment}
\usepackage{rotating}

\DeclareMathOperator*{\argmax}{arg\,max}
\DeclareMathOperator*{\argmin}{arg\,min}

\newtheorem{theorem}{Теорема}
\newtheorem{lemma}[theorem]{Лемма}
\newtheorem{definition}{Определение}[section]

\numberwithin{equation}{section}

\newcommand*{\No}{No.}
\begin{document}

\title{\bf Смесь экспертов\thanks{Работа выполнена при поддержке РФФИ и правительства РФ.}}
\date{}
\author{}
\maketitle

\begin{center}
\bf
А.\,В.~Грабовой\footnote{Московский физико-технический институт, grabovoy.av@phystech.edu}, В.\,В.~Стрижов\footnote{Московский физико-технический институт, strijov@ccas.ru}

\end{center}

{\centering\begin{quote}
\textbf{Аннотация:} 


\smallskip
\textbf{Ключевые слова}: ; .

\smallskip
\textbf{DOI}: 00.00000/00000000000000
\end{quote}
}

\section{Введение}

\section{Постановка задачи}

Задана выборка:
\begin{equation}
\label{eq:st:1}
\begin{aligned}
\textbf{X} \in \mathbb{R}^{N \times n},
\end{aligned}
\end{equation}
где~$N$~---~количество объектов в выборке, а~$n$~---~размерность признакового пространства.

\subsection{Смесь экспертов}
\begin{definition}
Смесь экспертов~---~мультимодель, определяющая правдоподобие каждой $\pi_k$ каждой модели $\textbf{f}_k$ на объекте $\textbf{x}$ на основе его признакового опсиания.

\begin{equation}
\label{eq:st:2}
\begin{aligned}
\hat{\textbf{f}} = \sum_{k=1}^{K}\pi_{k}\textbf{f}_k, \qquad \pi_{k}\left(\textbf{x}, \textbf{V}\right):\mathbb{R}^{2\times n} \to [0, 1], \qquad \sum_{k=1}^{K}\pi_{k}\left(\textbf{x}, \textbf{V}\right) = 1
\end{aligned}
\end{equation}
где~$\textbf{f}$~---~мультимодель, а $\textbf{f}_k$ является некоторой моделью, $\pi_k$~---~параметрическая модель.
\end{definition}

Правдоподобие модели:
\begin{equation}
\label{eq:st:3}
\begin{aligned}
p\left(\textbf{y}, \textbf{Z}|\textbf{X}, \textbf{W}, \textbf{V}, \beta\right) &= \prod_{i=1}^{N}\prod_{k=1}^{K}\left[\pi_{k}\left(\textbf{x}_i,\textbf{V}\right)\text{N}\left(y_i|\textbf{w}_{k}^{\mathsf{T}}\textbf{x}_i, \beta^{-1}\right)\right]^{z_{ik}}
\end{aligned}
\end{equation}

Логарифм правдоподобия модели:
\begin{equation}
\label{eq:st:4}
\begin{aligned}
\log p\left(\textbf{y}, \textbf{Z}|\textbf{X}, \textbf{W}, \textbf{V}, \beta\right) = \sum_{i=1}^{N}\sum_{k=1}^{K}z_{ik}\left[\log\pi_k\left(\textbf{x}_i, \textbf{V}\right) - \frac{\beta}{2}\left(y_i - \textbf{w}_{k}^{\mathsf{T}}\textbf{x}_i\right)^{2}+\frac{1}{2}\left(\log\beta-\log2\pi\right)\right]
\end{aligned}
\end{equation}

\paragraph{E-step}~

Найдем $q\left(\textbf{Z}\right)$:
\begin{equation}
\label{eq:st:5}
\begin{aligned}
q\left(\textbf{Z}\right) &= \mathsf{E}_{q/\textbf{Z}}\log p\left(\textbf{y}, \textbf{Z}|\textbf{X}, \textbf{W}, \textbf{V}, \beta\right) \\
p\left(z_{ik} = 1\right) &= \frac{\exp\left(\log\pi_{k}\left(\textbf{x}_{i}, \textbf{V}\right) - \frac{\beta}{2}\left[y_{i} - \textbf{w}_{k}^{\mathsf{T}}\textbf{x}_{i}\right]^{2}\right)}{\sum_{k'=1}^{K}\exp\left(\log\pi_{k'}\left(\textbf{x}_{i}, \textbf{V}\right) - \frac{\beta}{2}\left[y_{i} - \textbf{w}_{k'}^{\mathsf{T}}\textbf{x}_{i}\right]^{2}\right)}
\end{aligned}
\end{equation}

\paragraph{M-step}~
\begin{equation}
\label{eq:st:6}
\begin{aligned}
\mathsf{E}_{q} p\left(\textbf{y}, \textbf{Z}|\textbf{X}, \textbf{W}, \textbf{V}, \beta\right) &= \mathcal{F}\left(\textbf{W}, \textbf{V}, \beta\right) \\
\mathcal{F}\left(\textbf{W}, \textbf{V}, \beta\right) &= \sum_{i=1}^{N}\sum_{k=1}^{K}\mathsf{E}z_{ik}\left[\log\pi_k\left(\textbf{x}_i, \textbf{V}\right) - \frac{\beta}{2}\left(y_i - \textbf{w}_{k}^{\mathsf{T}}\textbf{x}_i\right)^{2} +\frac{1}{2}\left(\log\beta-\log2\pi\right)\right]
\end{aligned}
\end{equation}

Найдем $\beta$:
\begin{equation}
\label{eq:st:7}
\begin{aligned}
\frac{\partial \mathcal{F}\left(\textbf{W}, \textbf{V}, \beta\right)}{\partial \beta} &= \frac{1}{2}\sum_{i=1}^{N}\sum_{k=1}^{K}\mathsf{E}z_{ik}\left[-\left(y_i - \textbf{w}_{k}^{\mathsf{T}}\textbf{x}_i\right)^{2}+\frac{1}{\beta}\right] = 0 \\
\frac{1}{\beta} &= \frac{1}{N}\sum_{i=1}^{N}\sum_{k=1}^{K}\left[y_{i} -\textbf{w}_{k}^{\mathsf{T}}\textbf{x}_{i}\right]^{2}\mathsf{E}z_{ik}
\end{aligned}
\end{equation}

Найдем $\textbf{W}$:
\begin{equation}
\label{eq:st:8}
\begin{aligned}
\frac{\partial \mathcal{F}\left(\textbf{W}, \textbf{V}, \beta\right)}{\partial \textbf{w}_k} &= -\beta\sum_{i=1}^{N}\mathsf{E}z_{ik}\left[-y_i\textbf{x}_i+\textbf{x}_i\textbf{x}_i^{\mathsf{T}}\textbf{w}_k\right] = 0 \\
\textbf{w}_k &= \left[\sum_{i=1}^{N}\textbf{x}_i\textbf{x}_{i}^{\mathsf{T}}\mathsf{E}z_{ik}\right]^{-1}\left[\sum_{i=1}^{N}y_i\mathsf{E}z_{ik}\textbf{x}_i\right]
\end{aligned}
\end{equation}

Найдем $\textbf{V}$:

Аналитически решение не ищется, поэтому воспользуемся градиентным спуском для максимизации правдоподобия модели:
\begin{equation}
\label{eq:st:9}
\begin{aligned}
\textbf{V}^{j+1} &= \textbf{V}^{j} + \alpha\frac{\partial \mathcal{F}\left(\textbf{W}, \textbf{V}^{j}, \beta\right)}{\partial \textbf{V}} 
\end{aligned}
\end{equation}

\section{Вычислительный эксперимент}

\section{Заключение}

\begin{thebibliography}{99}

\end{thebibliography}

\end{document}

